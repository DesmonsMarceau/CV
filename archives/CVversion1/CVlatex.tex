\documentclass[11pt,a4paper,sans]{moderncv}
\moderncvstyle{casual}
\moderncvcolor{blue}
\setlength{\hintscolumnwidth}{2cm}
\usepackage[utf8]{inputenc}
\usepackage[scale=0.8]{geometry}
\usepackage{helvet}
\usepackage[french]{babel}
\name{Marceau}{Desmons}
\title{titre complémentaire}
\address{4, rue des 3 journées}{76\,100 Rouen}{France}
\phone[mobile]{06~27~46~40~40}
\phone[fixed]{02~35~55~82~57}
\phone[fax]{02~35~55~82~57}
\email{marceau.desmons@gmail.com}
\social[linkedin]{marceau-desmons}
\extrainfo{informations complémentaires.}
\photo[64pt][0.4pt]{maphoto}
\quote{Encore un titre}

\begin{document}
\makecvtitle

\section{Formation}
\cventry{1999--2000}{Baccalauréat Série S}{Lycée Jean Moulin}{Brest}{\textit{mention Bien}}%
            {Option Sciences de l'ingénieur}% on peut mettre ici de 3 à 6 arguments qui peuvent être laissés vides
\cventry{2000--2005}{\'Ecole d'ingénieur}{Institut Supérieur de Génie \'Electronique}{Sens}{\textit{Ingénieur réseau}}{Description}
\section{Experience professionnelle}
\cventry{2005--2009}{Ingénieur de recherche}{Commissariat à l'énergie atomique}{Grenoble—Isère}%
            {recherche et développement sur des nano-membranes de Silicium.}%
{%
\begin{itemize}%
\item  Mise en œuvre et procédés en salle blanche ;
\item intégration et caractérisation des membranes MEMS d'épaisseur nanométrique
  \begin{itemize}%
  \item AFM ;
  \item  Vibromètre laser ;
  \item MEB.
  \end{itemize}
\end{itemize}}
\cventry{2010--2013}{Ingénieur d'études de projets}{Cerdux}{Reims}{}{%
\begin{itemize}
\item Études de développement d'installations ou de systèmes
industriels automatisés pour définir la solution optimale dans le
contrôle des mouvements des machines ;
\item Rédaction et suivi d'offres proposant des solutions techniques
selon les besoins du client.
\end{itemize}}
\section{Langues}
%Possibilité d'insérer des commentaires dans les entrées
\cvitemwithcomment{Anglais}{Lu, parlé, écrit}{un commentaire si besoin}
\cvitemwithcomment{Allemand}{Scolaire}{Idem}
\section{Compétences informatiques}
%possibilité de mettre les entrées en deux colonnes
\cvdoubleitem{Java}{blabla, blabla}{C++}{blabla, blabla}
\cvdoubleitem{Php}{blabla, blabla}{Pascal}{blabla, blabla}
\cvdoubleitem{\LaTeX}{blabla, blabla}{Python}{blabla, blabla}
\section{Centres d'intérêts}
\cvitem{hobby 1}{Description}
\cvitem{hobby 2}{Description}

\end{document}
